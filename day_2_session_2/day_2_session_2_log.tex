\documentclass[12pt]{beamer}
\usetheme{default} 

\setbeamertemplate{navigation symbols}{} %gets rid of navigation symbols
\setbeamertemplate{footline}{} %gets rid of bottom navigation bars
\setbeamertemplate{footline}[page number]{} %use this for page numbers

\setbeamertemplate{footline}{%
  \raisebox{5pt}{\makebox[\paperwidth]{\hfill\makebox[10pt]{\scriptsize\insertframenumber~~}}}}

\setbeamertemplate{itemize items}[circle] %round bullet points
\setlength\parskip{10pt} % white space between paragraphs

\usepackage{wrapfig}
\usepackage{subfig}
\usepackage{setspace}
\usepackage{enumerate}
\usepackage{graphicx}
\usepackage{amsmath}
\usepackage{amsfonts}
\usepackage{amssymb}
\usepackage{amsthm}
\usepackage[UKenglish]{isodate}
\usepackage{tikz}
\usepackage{pgfplots}
\usepackage{natbib}
\def\checkmark{\tikz\fill[scale=0.4](0,.35) -- (.25,0) -- (1,.7) -- (.25,.15) -- cycle;} 

% allow drawing arrows
\usetikzlibrary{arrows}
\tikzstyle{arrow}=[draw, -latex] 

% skips
\setlength{\abovecaptionskip}{15pt plus 3pt minus 2pt}
\setlength{\belowcaptionskip}{5pt plus 3pt minus 2pt}
% bracketing shortcuts
\newcommand{\paren}[1]{\left(#1\right)}
\newcommand{\sqbracket}[1]{\left[#1\right]}
\newcommand{\cbracket}[1]{\left\{#1\right\}}
\newcommand{\abs}[1]{\left\lvert#1\right\rvert}
\newcommand{\norm}[1]{\left\lVert#1\right\rVert}
% set up the argmin operator, argmax
\DeclareMathOperator*{\argmin}{arg\,min}
\DeclareMathOperator*{\argmax}{arg\,max}

\newcommand{\myframe}[1]{\begin{frame} \frametitle{#1}}
% the preamble
\title{Day 2, Session 2: Logs/Exponentiation}
\author{Brian D. Williamson}
\institute{EPI/BIOST Bootcamp 2016}
\date{26 September 2016}

% Start the document
\begin{document}
% The title page
\begin{frame}
\titlepage
\end{frame}

\section{Exponentiation}
\myframe{Exponentiation}
\begin{itemize}
\item A mathematical operation corresponding to repeated multiplication
\item[]
\item The second in the order of operations! (P{\textbf E}MDAS)
\item[]
\item Composed of two numbers: a base, $b$, and an exponent, $n$
\item[]
\item $b^n = \underbrace{b\times b \times \cdots \times b}_\text{$n$ times}$
\item[]
\item Represented as $b^n$ or as $b \wedge n$
\end{itemize}
\end{frame}

\myframe{Positive vs negative exponents}
\begin{itemize}
\item Exponents correspond to multiplication
\item[]
\item Positive exponent: multiplication, e.g. $2^2 = 2\times 2$
\item[]
\item Negative exponent: multiplication of reciprocals, i.e. $2^{-2} = \dfrac{1}{2^2} = \frac{1}{2}\times \frac{1}{2}$
\end{itemize}
\end{frame}

\myframe{Properties of exponents}
\begin{itemize}
\item For any base $b$ and any $n$ an integer:
\begin{itemize}
\item $b^0 = 1$
\item[]
\item $b^1 = b$
\item[]
\item $b^{n+1} = b^n \times b$
\end{itemize}
\item[]
\item For $b \neq 0$ and any $n$ an integer:
\begin{itemize}
\item $b^n = b^{n+1}/b$
\item[]
\item $b^{-n} = 1/b^n$
\end{itemize}
\end{itemize}
\end{frame}

\myframe{Exponent identities}
\begin{itemize}
\item For all $b,c \neq 0$:
\begin{itemize}
\item $b^{m+n} = b^m \times b^n$
\item[]
\item $b^{m \times n} = (b^m)^n$
\item[]
\item $(b\times c)^n = b^n \times c^n$
\end{itemize}
\end{itemize}
\end{frame}

\myframe{Example: integer exponent properties and identities}
\begin{itemize}
\item Take $b = 2$
\item[]
\item $2^0 = 1$
\item[]
\item $2^1 = 2$
\item[]
\item $2^2 = 2 \times 2$
\item[]
\item $2^3 = 2^2 \times 2 = 8$
\item[]
\item $(2 \times 3)^2 = 2^2 \times 3^2 = 4 \times 9 = 36$ (check: $2 \times 3 = 6, 6^2 = 36$)
\end{itemize}
\end{frame}

\myframe{Rational exponents (roots)}
\begin{itemize}
\item $n$th root of $b$: the number $x$ such that $x^n = b$ 
\item[]
\item Written as $b^{1/n}$ or $\sqrt[n]{b}$
\item[]
\item Some identities (for $b$ positive):
\begin{itemize}
\item $b = (b^n)^{1/n}$
\item[]
\item $b^{m/n} = (b^m)^{1/n} = \sqrt[n]{b^m}$
\end{itemize}
\item[]
\item Example: $\sqrt{1/(36 x^2)} = \dfrac{\sqrt{1}}{\sqrt{36x^2}} = \dfrac{1}{\sqrt{36}\sqrt{x^2}} = \dfrac{1}{6x}$
\end{itemize}
\end{frame}

\myframe{Exponential function}
\begin{itemize}
\item An important constant: $e$, approximately $2.718$
\item[]
\item Useful as a base for powers
\item[]
\item Define $\exp(x) = e^x$
\item[]
\item Useful identity: $\exp(x+y) = e^{x+y} =  \exp(x)\times \exp(y)$
\end{itemize}
\end{frame}

\myframe{Exercise: exponents and the exponential function}
\begin{enumerate}
\item What is the result of $x^2$ multiplied by $x^3$?
\item[]
\item $(x^{-2})^4 = $?
\item[]
\item $\exp(x - y) = $?
\end{enumerate}
\end{frame}

\myframe{Solutions: exponents and the exponential function}
\begin{enumerate}
\item $x^2 \times x^3 = x^5$, since we add the exponents when we multiply
\item[]
\item $(x^{-2})^4 = x^{-2 \times 4} = x^{-8}$
\item[]
\item $\exp(x - y) = e^{x - y} = e^x \times e^{-y} = e^x/e^y = \exp(x)/\exp(y)$
\end{enumerate}
\end{frame}

\section{Logs}
\myframe{Logarithms}
\begin{itemize}
\item Exponents correspond to multiplication
\item[]
\item Addition is easier than multiplication
\item[]
\item Logarithms (logs) transform multiplication into addition!
\end{itemize}
\end{frame}

\myframe{Logs: definition}
\begin{itemize}
\item Defined as the inverse operation of exponentiation
\item[]
\item Takes a base $b$ and a number $x$
\item[]
\item The log of $x$ to base $b$ is the number $y$ such that $b^y = x$
\item[]
\item Written as $\log_b(x) = y$
\item[]
\item Natural log: $\log_e(x)$, commonly written $\log(x)$
\end{itemize}
\end{frame}

\myframe{Logs: definition}
\begin{itemize}
\item Undefined for $x \leq 0$
\item[]
\item Log is an increasing function: as $x$ increases, $\log_b(x)$  increases
\item[]
\item $\log_b(b) = 1$
\end{itemize}
\end{frame}

\myframe{Logs: identities}
\begin{itemize}
\item Multiplication: $\log_b(xy) = \log_b(x) + \log_b(y)$
\item[]
\item Division: for $y \neq 0$, $\log_b(x/y) = \log_b(x) - \log_b(y)$
\item[]
\item Powers: $\log_b(x^p) = p \log_b(x)$
\item[]
\item Roots: for $p \neq 0$, $\log_b(x^{1/p}) = \log_b(x)/p$
\item[]
\item Inverse function: $\log_b(b^x) = x\log_b(b) = x$
\end{itemize}
\end{frame}

\myframe{Example: log identities}
\begin{itemize}
\item Multiplication: $\log (2 \times 3) = \log (2) + \log (3) = \log (6)$
\item[]
\item Logs of numbers $< 1$ are negative: $\log (2/3) = \log (2) - \log (3) < 0$
\item[]
\item Power: $\log (x^2) = 2\log (x)$
\end{itemize}
\end{frame}

\myframe{Common bases, changing base}
\begin{itemize}
\item The three most common bases: $e$, $10$, and $2$
\item[]
\item $e$ --- common in mathematics
\item[]
\item 10 --- common for calculating numbers in the decimal system
\item[]
\item 2 --- common in computer science
\item[]
\item Changing between bases: $\log_b(x) = \dfrac{\log_k(x)}{\log_k(b)}$
\end{itemize}
\end{frame}

\myframe{$\exp(\cdot)$ and $\log(\cdot)$}
\begin{itemize}
\item Recall $\exp(x) = e^x$
\item[]
\item Natural log: $\log (x) = \log_e(x)$
\item[]
\item So $x = \log [\exp(x)]$! And $x = \exp[\log(x)]$!
\end{itemize}
\end{frame}

\myframe{Exercise: logarithms}
\begin{enumerate}
\item $\log(xy) = $?
\item[]
\item $\log(x/y) = $?
\item[]
\item $\log[\exp(2x)] = $?
\item[]
\item $\exp[\log(x^2)] = $?
\end{enumerate}
\end{frame}

\myframe{Solutions: logarithms}
\begin{enumerate}
\item $\log(xy) = \log(x) + \log(y)$
\item[]
\item $\log(x/y) = \log(x) - \log(y)$
\item[]
\item $\log[\exp(2x)] = 2x$
\item[]
\item $\exp[\log(x^2)] = \exp[2\log(x)] = \exp(2)\exp[\log(x)] = x\exp(2)$
\end{enumerate}
\end{frame}

\myframe{Uses of logarithms in statistics}
\begin{itemize}
\item Transformation of the data --- look at a multiplicative relationship rather than an additive relationship
\item[]
\item Logistic regression, Poisson regression
\item[]
\item For more, see BIOST 512/513!
\end{itemize}
\end{frame}

\myframe{Summary}
\begin{itemize}
\item Exponentiation: can create terms of higher order (larger exponent) than linear terms (exponent 1)
\item[]
\item Logarithms: turn multiplication into addition, using a base
\item[]
\item Most common base: $e$
\item[]
\item Useful for transforming data or different types of regression (logistic, Poisson)
\end{itemize}
\end{frame}
\end{document}
