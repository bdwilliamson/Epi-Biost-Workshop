\documentclass[11pt]{beamer}
\usetheme{default} 

\setbeamertemplate{navigation symbols}{} %gets rid of navigation symbols
\setbeamertemplate{footline}{} %gets rid of bottom navigation bars
\setbeamertemplate{footline}[page number]{} %use this for page numbers

\setbeamertemplate{footline}{%
  \raisebox{5pt}{\makebox[\paperwidth]{\hfill\makebox[10pt]{\scriptsize\insertframenumber~~}}}}

\setbeamertemplate{itemize items}[circle] %round bullet points
\setlength\parskip{10pt} % white space between paragraphs

\usepackage{wrapfig}
\usepackage{subfig}
\usepackage{setspace}
\usepackage{enumerate}
\usepackage{graphicx}
\usepackage{amsmath}
\usepackage{amsfonts}
\usepackage{amssymb}
\usepackage{amsthm}
\usepackage[UKenglish]{isodate}
\usepackage{tikz}
\usepackage{pgfplots}
\usepackage{natbib}
\usepackage{hyperref}
\hypersetup{
    colorlinks=true, 
    urlcolor=blue
}
\def\checkmark{\tikz\fill[scale=0.4](0,.35) -- (.25,0) -- (1,.7) -- (.25,.15) -- cycle;} 

% allow drawing arrows
\usetikzlibrary{arrows}
\tikzstyle{arrow}=[draw, -latex] 

% bracketing shortcuts
\newcommand{\paren}[1]{\left(#1\right)}
\newcommand{\sqbracket}[1]{\left[#1\right]}
\newcommand{\cbracket}[1]{\left\{#1\right\}}
\newcommand{\abs}[1]{\left\lvert#1\right\rvert}
\newcommand{\norm}[1]{\left\lVert#1\right\rVert}
% set up the argmin operator, argmax
\DeclareMathOperator*{\argmin}{arg\,min}
\DeclareMathOperator*{\argmax}{arg\,max}

\newcommand{\myframe}[1]{\begin{frame} \frametitle{#1}}

% New itemize environment, with spaces
\newenvironment{spaceitemize}
{ \begin{itemize}
    \setlength{\itemsep}{10pt}
    \setlength{\parskip}{0pt}
    \setlength{\parsep}{0pt}     }
{ \end{itemize}                  } 


% the preamble
\title{Day 3, Session 1: Installing R and RStudio}
\author{Jessica Williams-Nguyen and Brian D. Williamson}
\institute{EPI/BIOST Bootcamp 2018}
\date{25 September 2018}

% Start the document
\begin{document}
% The title page
\begin{frame}
\titlepage
\end{frame}

\begin{frame}
\frametitle{Learning objectives}
By the end of this session, you should be able to
\begin{itemize}
\item \textbf{describe} some of the advantages of using R
\item \textbf{describe} some of the advantages of using RStudio
\item \textbf{install} R and RStudio on your own computer
\end{itemize}
\end{frame}

\myframe{\includegraphics[width = 0.2\textwidth]{figs/Rlogo.png}}
R is a \textcolor{blue}{free}, \textcolor{green}{open source} software package that can be used for data analysis, graphics, and programming. \pause

At its core, R is an interactive, command-driven \textcolor{blue}{language}: you type a command and R executes it and returns results. \pause

While R is sometimes said to have a steep learning curve, it is relatively easy to get set up with the basics and analyze some data.
\end{frame}

\myframe{Why R?}
R has many advantages, including: \pause
\begin{spaceitemize}
\item Free and open source! \pause
\item Active group of contributors (anyone!) \pause
\item Flexible \pause
\item Large set of packages for data analysis \pause
\end{spaceitemize}

However, this comes with some challenges: \pause
\begin{spaceitemize}
\item Sometimes packages don't do what they say they do... \pause
\begin{itemize}
\item ...but you can trust basically anything written by the R Core Team, the \href{https://www.rstudio.com/products/rpackages/}{RStudio Team}, \href{http://hadley.nz/}{Hadley Wickham}, or \href{https://github.com/yihui}{Yihui Xie}
\end{itemize}
\end{spaceitemize}
\end{frame}


\myframe{\includegraphics[width = 0.2\textwidth]{figs/RStudiologo.png}}
RStudio is both an integrated development environment (IDE) and a graphical user interface (GUI) for R programming and data analysis. The free version is also open source. \pause

It includes a console, text editor that allows for direct execution of code, as well as tools for importing/exporting data, plotting, file management, and debugging. (We will cover all of these terms later!)
\end{frame}

\myframe{Why RStudio?}
The base R GUI is both light and functional. Sometimes, however, we want more than that!

RStudio adds: \pause
\begin{spaceitemize}
\item An improved layout of the console, text editor, and other tools \pause
\item Support for embedding reproducible research tools \pause
\item Support for building your own R packages \pause
\item Integrated R help and documentation \pause
\item A pretty good debugger \pause
\end{spaceitemize}

The only downside to RStudio is that it takes a decent amount of memory to run... but for most purposes, this isn't a problem. 
\end{frame}


\myframe{Why two programs?}
R is a programming software, prepackaged with a GUI. However, R programs can be executed from the command line without an interactive interface. \pause

RStudio is a GUI, and is a helpful tool for working in R. Using RStudio makes it easier to: \pause
\begin{spaceitemize}
\item write R scripts to save your work, along with comments for what your code does \pause
\item write reports with code embedded (using Rmarkdown) \pause
\item organize your data analysis workflow (e.g., reading in data, access help files)
\end{spaceitemize}
\end{frame}

\myframe{Why two programs?}
At the end of the day: you are executing commands/programs in R, but using RStudio as an intuitive interface to the software (much like your operating system is a GUI to the machine language that your computer understands). \pause

The combination of R and RStudio makes reproducible research attainable by everyday users. The RStudio environment has many easy options to facilitate this through R, and a lot of support behind the scenes.


\end{frame}

\myframe{Installing R}
\begin{enumerate}
\item Go to \url{https://cran.r-project.org/}
\item[] \pause
\item Choose the correct link under \textcolor{blue}{\underline{\texttt{Download and Install R}}} \pause
\begin{spaceitemize}
\item Windows users, select \textcolor{blue}{\underline{\texttt{install R for the first time}}}
\item Mac users, click the \textcolor{blue}{\underline{\texttt{R-[replace with most recent version number].pkg}}} link and install
\item Linux users, I assume you know what you are doing
\end{spaceitemize}
\end{enumerate}

\end{frame}

\myframe{Installing RStudio}
\begin{enumerate}
\item Go to \url{https://www.rstudio.com/}
\item[] \pause
\item Scroll down until you see the headers for \textcolor{blue}{\texttt{RStudio}}, \textcolor{blue}{\texttt{Shiny}}, and \textcolor{blue}{\texttt{R Packages}}  (see figure) \pause
\item[]
\item Click \textcolor{blue}{\underline{Download}}
\item[] \pause
\item Click the green download button in the column for \texttt{RStudio Desktop}
\item[] \pause
\item Choose the correct installer for your operating system and click the link
\end{enumerate}
\centering
\includegraphics[width = 0.5\textwidth]{figs/rstudio_download.png}
\end{frame}
\end{document}